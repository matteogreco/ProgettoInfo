\documentclass[a4paper, 12pt]{article}
\author{Matteo Greco \& Nicol\`o Grassi}
\title{Progetto di Informatica}
\date{}
\begin{document}
\maketitle
\begin{abstract}
Siamo Matteo e Nicol\`o, due studenti di Ingegneria Biomedica del Politecnico di Milano.\\\\Il nostro professore di informatica Marco D. Santambrogio ci sta dando la possibilit\`a di svolgere, sotto la sua supervisione, un progetto che, oltre a servire come valutazione finale per la parte di informatica, pensiamo possa essere un'occasione per poter misurare le nostre abilit\'a/competenze nell'affrontare i problemi.\\\\
Il progetto si basa sull'utilizzo dell'informatica come mezzo per lo sviluppo di nuove tecnologie che interessino pi\`u direttamente il nostro corso di studio.
Il tema del progetto deriva dall'interesse comune per il sempre pi\`u ampio mercato degli smart-band, accessorio che noi vogliamo rendere di 'vitale' importanza. E' da qui che \`e nato il nostro concept: uno smart-band che consenta di monitorare 24/7 i principali parametri vitali della persona che lo indossa. Vi verr\`a da pensare che possa essere una cosa gi\`a vista, dato che smart-band che registrano la frequenza cardiaca sono in commercio da tempo; ma il nostro ha una marcia in pi\`u: oltre a poter monitorare il battito, esso \`e in grado, attraverso un sensore wireless associato, di tracciare, registrare e interpretare l'ECG. 
\end{abstract}
\end{document}